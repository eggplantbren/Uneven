% Every Latex document starts with a documentclass command
\documentclass[a4paper, 10pt]{article}

% Load some packages
\usepackage{graphicx} % This allows you to put figures in
\usepackage{natbib}   % This allows for relatively pain-free reference lists
\usepackage[left=3cm,top=3cm,right=3cm]{geometry} % The way I like the margins
\usepackage{dsfont}

% This helps with figure placement
\renewcommand{\topfraction}{0.85}
\renewcommand{\textfraction}{0.1}
\parindent=0cm

% Set values so you can have a title
\title{}
\author{Me}
\date{\today}

% Document starts here
\begin{document}

% Actually put the title in
\maketitle

%\abstract{This is the abstract}

The unknowns are the sine and cosine amplitudes $A$ and $B$ respectively,
and the frequency $\nu$. I.e. the time variations are of the form

\begin{eqnarray}
m(t) &=& A\sin(2\pi\nu t) + B\cos(2\pi\nu t)\label{eq:sinusoids}
\end{eqnarray}

The following derivation follows those by Larry Bretthorst. Writing
Equation~\ref{eq:sinusoids} in the more general form:
\begin{eqnarray}
m(t) &=& \sum_{j=1}^N A_j f_j(t; \nu)\label{eq:sinusoids}
\end{eqnarray}

The data are the measurements
$\mathbf{y} = \{y_i\}_{i=1}^n$.
The likelihood is
\begin{eqnarray}
p(\mathbf{y} | A, B, \nu)
&=&
\prod_{i=1}^{n} \frac{1}{\sigma_i \sqrt{2\pi}}
\exp\left\{
-\frac{1}{2\sigma_i^2}
\left[y_i - \sum_{j=1}^N A_j f_j(t_i; \nu))\right]^2
\right\}
\\
&\propto&
\exp\left\{
-\frac{1}{2}\sum_{i=1}^n
\left(
\frac{y_i^2}{\sigma_i^2} - 2\frac{y_i \sum_{j=1}^N A_j f_j(t_i; \nu)}{\sigma_i^2}
+ \left(\sum_{j=1}^N A_j f_j(t_i; \nu)\right)^2
\right)
\right\}\nonumber\\
&\propto&
\exp\left\{
(\mathbf{A}\cdot\mathbf{f})
 -\frac{1}{2} \mathbf{A^T}\mathbf{g}\mathbf{A}
\right\}
\end{eqnarray}

Where $\mathbf{f}$ is a vector of the dot products of the model functions
(sine, cosine) with the data, and $\mathbf{g}$ is the matrix of cross products
of the model functions with themselves. All dot products are taken with
by summing over $i$, and with respect to $1/\sigma_i^2$ weights.


\end{document}

